\begin{frame}{Selektion}
    \begin{columns}[T] % T aligns the tops of the columns
        \begin{column}{0.4\textwidth}
            \textbf{Elitismus:}
            \begin{itemize}
                \item Wahl der besten x-Prozent der Individuen
                \item Vorteil: schnelles Erreichen des (lokalen) Minimums
                \item Nachteil: eventuell globales Minimum schwerer zu erreichen
            \end{itemize}
        \end{column}
        \begin{column}{0.6\textwidth}
            \textbf{Glücksradauswahl (universelles Stitchprobenziehen):}
            \begin{itemize}
                \item Skalierung der Fitness, sodass die Summe der
                Fitnesswerte 1 ergibt
                \item hintereinander reihen der Individuen mit ihren Fitnesswerten
                \item Auswahl von Individuen in gleichmäßigen Abständen (angepasst an Selektionsrate)
                \item Vorteil: höhere Diversität
                \item Nachteil: langsamere Konvergenz
            \end{itemize}
        \end{column}
    \end{columns}
\end{frame}