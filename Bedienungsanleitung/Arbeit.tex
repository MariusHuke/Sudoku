\documentclass[fontsize=11pt, DIV=calc]{scrartcl} % Klasse der Bibliothek KOMAscript
\usepackage[T1]{fontenc}
\usepackage{enumerate}
\usepackage{hyperref}
\usepackage[utf8]{inputenc}
\usepackage[ngerman]{babel} % typografische Regeln für die neue deutsche Rechtschreibung
\usepackage{float}
\usepackage{listings,lstautogobble} % Code-Listings
\usepackage{lmodern} % verändert die Schriftart
\usepackage{enumitem} % ermöglicht die Anpassung von Aufzählungszeichen
\usepackage{geometry} % kann die Maße eines Dokumentes festlegen. Hier nur für Titelseite genutzt. Um alle anderen Maße kümmert sich KOMAscript!
\usepackage{graphicx} % zeigt Bilder schon vor der Compillierung
\usepackage{url}
\usepackage{xcolor} % kann Textfarben festlegen
\usepackage{csquotes}

\usepackage{amssymb} %mathematische Schreibweisen
\usepackage{amsmath}

\usepackage{biblatex} % kann Literaturverzeichnis erstellen (chem-angew) als alternative Zitationsweise

%%%%%%%%%%%%%%%Bibliography%%%%%%%%%%%%%%%%%%
%\addbibresource{Ausarbeitung/bibliography.bib} % Bibliographie
%%%%%%%%%%%%%%%Bibliography%%%%%%%%%%%%%%%%%%


\recalctypearea % berechnet den Satzspiegel neu. Muss am Ende der Präambel stehen. Neue Pakete bitte oberhalb dieser Zeile eintragen (dieser Befehl gehört zu KOMAscript bzw. dessen typearea-Paket)

%%%%%%%%%%%%%%%Colouring, Abstände usw.%%%%%%%%%%%%%%%%
\setlength{\parindent}{0.5cm}
\topmargin -1.0cm
\oddsidemargin 0cm
\evensidemargin 0cm
\setlength{\textheight}{23.37cm}
\setlength{\textwidth}{15.24cm}
\definecolor{codegreen}{rgb}{0,0.6,0}
\definecolor{codegray}{rgb}{0.5,0.5,0.5}
\definecolor{codepurple}{rgb}{0.58,0,0.82}
\definecolor{backcolour}{rgb}{0.95,0.95,0.92}
\setlength{\abovecaptionskip}{1pt} % Chosen fairly arbitrarily
\lstdefinestyle{mystyle}{ % Definition des Styles für Listings
    backgroundcolor=\color{white},   %replaced with white instead of backcolour
    commentstyle=\color{codegreen},
    keywordstyle=\color{blue},
    numberstyle=\tiny\color{codegray},
    stringstyle=\color{codepurple},
    basicstyle=\ttfamily\footnotesize,
    breakatwhitespace=false,         
    breaklines=true,                 
    captionpos=b,                    
    keepspaces=true,                 
    numbers=left,                    
    numbersep=5pt,                  
    showspaces=false,                
    showstringspaces=false,
    showtabs=false,                  
    tabsize=2,
    autogobble=true,
}
\lstset{style=mystyle}

%%%%%%%%%%%%%%%%Titel%%%%%%%%%%%%%%%%%%%%%
\hypersetup{
	pdftitle={PDFTitle},
	pdfauthor={Huke, Marius},
	pdfborder={0 0 0},
    colorlinks=true,
    linkcolor=black,
    urlcolor=blue,
    citecolor=black,
}

% TODO-Command {\todo}
% Arg 1: Text der angezeigt werden soll
\newcommand{\todo}[1]{\textcolor{red}{\textbf{TODO:} #1}}
% Section -Referenz
% Arg 1: Farbe des Links (optional, default: cyan)
% Arg 2: Label der zu verlinkenden Section
% Arg 3: Text des Links
\newcommand{\colorref}[3][cyan]{\hyperref[#2]{\textcolor{#1}{#3}}}

% Listing-Referenz
% Arg 1: Label des zu verlinkenden listings
% Arg 2: Text des Links
\newcommand{\nameref}[2]{\hyperref[#1]{\underline{#2}}}


\begin{document}
    %%%%%%%%%%%%%%%%%%%%%Input of sections%%%%%%%%%%%%%%%%%%%%%%%%
    \section*{Bedienungsanleitung}
    \subsection*{Startbildschirm}
    \begin{itemize}
        \item Zu Beginn Abfrage von Populationsgröße und Auswahl des Sudokus (diese können mit zugehörigen Nummern in Sudokus.txt gefunden werden)
        \item zugelassen sind Zahlen \(>1\) für Populationsgröße (Empfehlenswert: \([100,1000]\)) und \([0,39]\) für die Sudokus
        \item Nach "{}Ok{}"{} oder betätigen der Entertaste öffnet sich Fenster Löser
    \end{itemize}
    \subsection*{Löser}
    \begin{itemize}
        \item Links: 
        \begin{itemize}
            \item das beste Sudoku der momentanen Population (zu Beginn der Initialpopulation)
            \item grün \(\rightarrow\) gegebende Felder \\
            rot \(\rightarrow\) eingesetzte Felder mit Kollision \\
            weiß \(\rightarrow\) eingesetzte Felder ohne Kollision
        \end{itemize}
        \item Mitte:
        \begin{itemize}
            \item momentane Generation
            \item Summe der Kollisionen des besten Individuums (auch links gezeigt)
            \item Eingabefeld mit Generationen pro Schritt (Wie viele Generationen sollen in einem Schritt berechnet werden?)
            \\ \textbf{Achtung:} Im Programm ist außer der Lösung kein Abbruchkriterium festgelegt.\\ \(\rightarrow\) (empfohlen max. \(25-40\) Generationen)
            \item Eingabefeld mit Selektionsrate (empfohlen 20): Welcher Anteil der Population soll selektiert werden?
            \item Knopf Schritt: Berechnet die gegebende Anzah an Generationen
        \end{itemize}
        \item Rechts:
        \begin{itemize}
            \item Graph mit Best- und Durchschnittswert der Population (wird während Laufzeit erweitert)
            \item Reset: Bringt den Nutzer zurück in den Startbildschirm
            \item 0. Gen: Setzt zurück, erstellt neue Initialpopulation, behält aber Populationsgröße und Sudoku bei
        \end{itemize}
    \end{itemize}
\end{document}